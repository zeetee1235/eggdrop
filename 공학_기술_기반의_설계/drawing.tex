\documentclass[a4paper,12pt]{article}
\usepackage{CJKutf8}
\usepackage{amsmath}
\usepackage{geometry}
\usepackage{array}
\usepackage{booktabs}
\usepackage{enumitem}
\geometry{margin=2.5cm}

\title{\Large\textbf{에그드롭 구조물 설계서}}
\author{공학 기술 기반 설계}
\date{2025년 10월 22일}

\begin{document}
\begin{CJK}{UTF8}{mj}

\maketitle

\begin{center}
\large
정사면체 기반 충격 흡수 구조물
\end{center}

\vspace{1cm}

\tableofcontents
\newpage

% ============================================
\section{설계 개요}
% ============================================

\subsection{설계 목표}

본 구조물은 높은 곳에서 떨어뜨린 계란을 보호하기 위한 충격 흡수 장치입니다. 
정사면체의 구조적 안정성과 연장된 봉의 충격 분산 효과를 결합하여 설계했습니다.

\subsection{핵심 아이디어}

\begin{enumerate}[leftmargin=*]
    \item \textbf{정사면체 중심 구조}: 계란을 중앙에 배치하고 4개 꼭짓점으로 모든 방향의 충격 분산
    \item \textbf{외부 연장 봉}: 각 변을 3.5배로 연장하여 충격 지점과 계란 사이 거리 확보
    \item \textbf{내부 보강재}: 12개 지지대로 구조 강성 확보 및 변형 방지
    \item \textbf{테이프 면}: 24개 충격 흡수면이 에너지를 소산
\end{enumerate}

% ============================================
\section{구조 상세}
% ============================================

\subsection{기본 치수}

\begin{center}
\begin{tabular}{ll}
\toprule
\textbf{항목} & \textbf{치수} \\
\midrule
정사면체 한 변의 길이 & 80 mm \\
정사면체 높이 & 약 65.3 mm \\
외부 연장 봉 총 길이 & 280 mm (= 80 mm $\times$ 3.5) \\
외부 봉 한쪽 연장 길이 & 140 mm (중점 기준) \\
구조물 전체 크기 & 약 350 mm $\times$ 350 mm $\times$ 350 mm \\
\bottomrule
\end{tabular}
\end{center}

\subsection{정사면체 좌표}

중심 정사면체는 4개 꼭짓점으로 구성됩니다:

\begin{itemize}
    \item V0: (0, 0, 0) mm
    \item V1: (80, 0, 0) mm
    \item V2: (40, 69.3, 0) mm
    \item V3: (40, 23.1, 65.3) mm
\end{itemize}

이 4개 점을 연결하면 6개 변(모서리)이 만들어집니다:
V0-V1, V1-V2, V2-V0, V0-V3, V1-V3, V2-V3

% ============================================
\section{부품 구성}
% ============================================

\subsection{1. 정사면체 중심부}

\textbf{기능}: 계란을 보호하는 핵심 구조

\begin{itemize}
    \item 부품: 80 mm 길이 봉 6개
    \item 직경: 4 mm
    \item 재질: 나무젓가락
    \item 역할: 계란 고정 및 1차 충격 분산
\end{itemize}

\subsection{2. 외부 연장 봉}

\textbf{기능}: 충격 지점과 계란 사이 거리 확보, 봉 변형으로 에너지 흡수

\begin{itemize}
    \item 부품: 280 mm 길이 봉 6개
    \item 직경: 4 mm
    \item 재질: 플라스틱 봉
    \item 배치: 각 정사면체 변의 중점을 기준으로 양쪽 140 mm씩 연장
    \item 역할: 주 충격 흡수 (봉이 휘어지며 충격 완화)
\end{itemize}

\textbf{예시}: V0-V1 변(80mm)의 중점을 M이라 하면:
\begin{itemize}
    \item 외부 봉은 M에서 V0 방향으로 140 mm 연장
    \item M에서 V1 방향으로 140 mm 연장
    \item 총 280 mm 봉이 V0-V1 변과 일직선상에 배치
\end{itemize}

\subsection{3. 내부 지지대}

\textbf{기능}: 구조 강성 유지 및 하중 분산

\begin{itemize}
    \item 부품: 12개 봉
    \item 직경: 3 mm
    \item 재질: 나무젓가락
    \item 연결: 각 변의 중점 → 반대편 2개 꼭짓점
\end{itemize}

\textbf{연결 예시}:
\begin{itemize}
    \item V0-V1 변의 중점 M01 → V2
    \item V0-V1 변의 중점 M01 → V3
    \item (나머지 5개 변도 동일: 총 6 $\times$ 2 = 12개)
\end{itemize}

\subsection{4. 테이프 충격 흡수면}

\textbf{기능}: 종이/천이 찢어지며 충격 에너지 소산

\begin{itemize}
    \item 형태: 사다리꼴 24개
    \item 위치: 외부 봉 양 끝 30 mm 구간
    \item 재질: 얇은 종이 또는 천 (테이프로 고정)
    \item 구성: 
    \begin{itemize}
        \item 정사면체 꼭짓점 쪽: 12개
        \item 반대편 끝: 12개
    \end{itemize}
\end{itemize}

% ============================================
\section{재료 목록}
% ============================================

\begin{center}
\begin{tabular}{lccc}
\toprule
\textbf{부품명} & \textbf{규격} & \textbf{수량} & \textbf{재질} \\
\midrule
정사면체 변 & 80 mm, 직경 4 mm & 6개 & 나무젓가락 \\
외부 연장 봉 & 280 mm, 직경 4 mm & 6개 & 플라스틱 봉 \\
내부 지지대 & 길이 가변, 직경 3 mm & 12개 & 나무젓가락 \\
테이프 면 & 폭 30 mm & 24개 & 종이/천 \\
접착제 & - & 적량 & 글루건/목공풀 \\
고정 테이프 & - & 적량 & 종이테이프 \\
\bottomrule
\end{tabular}
\end{center}

\vspace{0.5cm}

\textbf{참고사항}:
\begin{itemize}
    \item 나무젓가락은 구하기 쉽고 가공이 용이합니다
    \item 플라스틱 봉은 적당한 탄성으로 충격 흡수에 효과적입니다
    \item 테이프 면은 얇을수록 찢어지며 에너지를 더 잘 흡수합니다
    \item 글루건은 빠른 접착에 유리하고, 목공풀은 강도가 높습니다
\end{itemize}

% ============================================
\section{조립 순서}
% ============================================

\subsection{단계 1: 정사면체 제작}

\begin{enumerate}
    \item 80 mm 봉 6개 준비
    \item 4개 꼭짓점을 만들도록 6개 변 연결
    \item 글루건으로 모든 접합부 고정
    \item 구조가 흔들리지 않는지 확인
\end{enumerate}

\textbf{팁}: 바닥(V0, V1, V2)부터 삼각형으로 만든 후 꼭대기(V3) 연결

\subsection{단계 2: 외부 연장 봉 부착}

\begin{enumerate}
    \item 각 변(80mm)의 중점 위치 표시
    \item 280 mm 봉을 중점 기준 양쪽 140 mm씩 배치
    \item 정사면체 변과 일직선이 되도록 정렬
    \item 중점 부근을 글루건으로 단단히 고정
    \item 총 6개 외부 봉 부착
\end{enumerate}

\textbf{주의}: 봉이 변과 정확히 일직선을 이루어야 충격 분산이 효과적입니다

\subsection{단계 3: 내부 지지대 설치}

\begin{enumerate}
    \item 각 변의 중점 위치 재확인
    \item 중점에서 반대편 2개 꼭짓점까지 거리 측정
    \item 해당 길이로 봉 자르기 (총 12개)
    \item 각 중점에서 반대편 꼭짓점으로 지지대 연결
    \item 모든 접합부 글루건으로 고정
\end{enumerate}

\textbf{예시}:
\begin{itemize}
    \item V0-V1 변 중점 → V2 연결
    \item V0-V1 변 중점 → V3 연결
    \item (5개 변 × 2 = 10개 더 연결)
\end{itemize}

\subsection{단계 4: 테이프 면 부착}

\begin{enumerate}
    \item 외부 봉 끝에서 30 mm 구간 표시
    \item 인접한 2개 봉의 끝점을 종이/천으로 연결
    \item 사다리꼴 형태로 자르기
    \item 테이프로 봉에 고정
    \item 한쪽 끝 12개, 반대편 끝 12개 총 24개 부착
\end{enumerate}

\textbf{팁}: 종이는 얇을수록 좋고, 너무 단단하지 않아야 찢어지며 충격 흡수

\subsection{단계 5: 마무리}

\begin{enumerate}
    \item 모든 접합부 재점검
    \item 흔들림 없이 단단한지 확인
    \item 계란 고정 위치 확인 (정사면체 중심)
    \item 가벼운 충격 테스트 (낮은 높이)
\end{enumerate}

% ============================================
\section{충격 흡수 원리}
% ============================================

\subsection{3단계 충격 흡수 시스템}

\textbf{1단계: 외부 봉 변형}
\begin{itemize}
    \item 280 mm 긴 봉이 충격을 받으면 휘어짐
    \item 휘어지는 과정에서 운동에너지 → 변형에너지로 전환
    \item 충격력의 70-80\% 1차 흡수
\end{itemize}

\textbf{2단계: 테이프 면 파손}
\begin{itemize}
    \item 종이/천 재질이 찢어지며 에너지 소산
    \item 찢어지는 과정 = 에너지 방출
    \item 충격력의 10-15\% 추가 흡수
\end{itemize}

\textbf{3단계: 내부 지지대 분산}
\begin{itemize}
    \item 12개 지지대가 하중을 여러 방향으로 분산
    \item 정사면체 형태 유지로 계란 직접 충격 방지
    \item 나머지 충격력 최종 분산
\end{itemize}

\subsection{작동 흐름도}

\begin{center}
\fbox{
\begin{minipage}{0.9\textwidth}
\centering
충격 발생 
\\ $\downarrow$ \\
외부 봉이 휘어지며 1차 충격 흡수 (70-80\%)
\\ $\downarrow$ \\
테이프 면이 찢어지며 2차 에너지 소산 (10-15\%)
\\ $\downarrow$ \\
내부 지지대가 하중을 분산 (나머지)
\\ $\downarrow$ \\
정사면체 중심부가 계란 보호
\\ $\downarrow$ \\
\textbf{계란 안전!}
\end{minipage}
}
\end{center}

% ============================================
\section{예상 성능}
% ============================================

\begin{center}
\begin{tabular}{ll}
\toprule
\textbf{항목} & \textbf{예상 값} \\
\midrule
목표 낙하 높이 & 3-5 m \\
구조물 무게 & 50-100 g \\
전체 크기 & 약 350 mm $\times$ 350 mm $\times$ 350 mm \\
충격 흡수율 & 85-95\% (3단계 합산) \\
제작 시간 & 약 2-3시간 \\
예상 비용 & 5,000-10,000원 \\
\bottomrule
\end{tabular}
\end{center}

\subsection{장점}

\begin{enumerate}
    \item \textbf{높은 충격 흡수율}: 3단계 시스템으로 효과적 보호
    \item \textbf{구조적 안정성}: 정사면체는 모든 방향 충격에 강함
    \item \textbf{가벼운 무게}: 나무젓가락과 플라스틱 봉 사용으로 경량화
    \item \textbf{제작 용이}: 복잡한 도구 없이 조립 가능
    \item \textbf{재료 구하기 쉬움}: 나무젓가락, 플라스틱 봉, 종이 등 구하기 쉬운 재료
\end{enumerate}

\subsection{주의사항}

\begin{enumerate}
    \item 외부 봉이 일직선을 이루지 않으면 충격 분산 효과 감소
    \item 테이프 면은 너무 두꺼우면 찢어지지 않아 효과 없음
    \item 접착이 약하면 충격시 구조물 분리 위험
    \item 계란은 반드시 정사면체 중심에 고정
\end{enumerate}

% ============================================
\section{수학적 계산}
% ============================================

\subsection{정사면체 기하학}

정사면체 한 변의 길이를 $a = 80$ mm라 할 때:

\textbf{높이}:
\[
h = \sqrt{\frac{2}{3}} \cdot a = \sqrt{\frac{2}{3}} \times 80 \approx 65.32 \text{ mm}
\]


\subsection{외부 봉 계산}

\textbf{총 길이}:
\[
L = 3.5 \times a = 3.5 \times 80 = 280 \text{ mm}
\]

\textbf{중점 기준 한쪽 연장}:
\[
\text{연장 길이} = \frac{L}{2} = \frac{280}{2} = 140 \text{ mm}
\]

\textbf{정사면체 변 대비 연장 부분}:
\[
\text{연장 = } \frac{(3.5 - 1) \times a}{2} = \frac{2.5 \times 80}{2} = 100 \text{ mm (한쪽)}
\]

% ============================================
\section{참고 자료}
% ============================================

\subsection{관련 물리 원리}

\begin{itemize}
    \item \textbf{충격량-운동량 정리}: $F \cdot \Delta t = \Delta p$
    \begin{itemize}
        \item 충격 시간을 늘리면 힘이 감소
        \item 봉이 휘어지는 시간 = 충격 시간 증가
    \end{itemize}
    
    \item \textbf{에너지 보존}: 운동에너지 → 변형에너지 + 열에너지
    \begin{itemize}
        \item 봉 변형, 종이 찢어짐으로 에너지 전환
    \end{itemize}
    
    \item \textbf{정사면체 구조}: 모든 방향에서 대칭적 지지
    \begin{itemize}
        \item 어느 방향 충격도 4개 꼭짓점이 분산
    \end{itemize}
\end{itemize}

\subsection{개선 아이디어}

\begin{enumerate}
    \item 외부 봉에 스펀지 감싸기 → 추가 충격 흡수
    \item 테이프 면을 2-3겹으로 → 다단계 파손으로 에너지 소산 증가
    \item 계란 주변에 솜 채우기 → 최종 보호층
    \item 낙하산 부착 → 낙하 속도 자체를 줄임
\end{enumerate}

\vspace{1cm}

\begin{center}
\large
\textbf{--- 설계서 끝 ---}
\end{center}

\end{CJK}
\end{document}
