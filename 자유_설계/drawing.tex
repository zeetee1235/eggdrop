\documentclass[a4paper,12pt]{article}
\usepackage{CJKutf8}
\usepackage{amsmath}
\usepackage{geometry}
\usepackage{array}
\usepackage{booktabs}
\usepackage{enumitem}
\geometry{margin=2.5cm}

\title{\Large\textbf{자유설계 계란낙하 보호구조물}}
\author{자유 설계}
\date{2025년 10월 23일}

\begin{document}
\begin{CJK}{UTF8}{mj}

\maketitle

\begin{center}
\large
고무줄 서스펜션 기반 탄성 충격 흡수 구조물
\end{center}

\vspace{1cm}

\tableofcontents
\newpage

% ============================================
\section{설계 개요}
% ============================================

\subsection{설계 목표}

본 구조물은 높은 곳에서 떨어뜨린 계란을 보호하기 위한 복합 충격 흡수 장치입니다. 
정육면체 프레임의 구조적 안정성과 고무줄 서스펜션의 탄성 복원력을 결합하여 설계했습니다.

\subsection{핵심 아이디어}

\begin{enumerate}[leftmargin=*]
    \item \textbf{정육면체 프레임}: 안정적인 기본 골격과 균등한 하중 분산
    \item \textbf{낙하산 시스템}: 낙하 속도를 줄여 충격 에너지 자체를 감소
    \item \textbf{12개 고무줄 서스펜션}: 계란을 중앙에 매달아 모든 방향 충격 흡수
    \item \textbf{3단계 보호층}: 지점토, 에어캡, 솜으로 계란 직접 보호
\end{enumerate}

% ============================================
\section{구조 상세}
% ============================================

\subsection{기본 치수}

\begin{center}
\begin{tabular}{ll}
\toprule
\textbf{항목} & \textbf{치수} \\
\midrule
정육면체 크기 & 400 mm $\times$ 400 mm $\times$ 400 mm \\
낙하산 직경 & 300-400 mm \\
고무줄 자연 길이 & 200 mm (각각) \\
보호층 총 두께 & 약 40-50 mm \\
구조물 전체 크기 & 약 600 mm $\times$ 600 mm $\times$ 600 mm \\
\bottomrule
\end{tabular}
\end{center}

\subsection{정육면체 구조}

400mm 정육면체는 12개 모서리와 8개 꼭짓점으로 구성됩니다:

\begin{itemize}
    \item 하면 4개 모서리 (바닥면)
    \item 상면 4개 모서리 (윗면)
    \item 수직 4개 모서리 (기둥)
    \item 8개 꼭짓점에서 12개 고무줄 연결점 제공
\end{itemize}

% ============================================
\section{부품 구성}
% ============================================

\subsection{1. 정육면체 프레임}

\textbf{기능}: 전체 구조의 기본 골격 역할

\begin{itemize}
    \item 부품: 400 mm 길이 막대 12개
    \item 직경: 5-8 mm
    \item 재질: 플라스틱 막대 또는 나무 막대
    \item 역할: 구조적 안정성 확보, 고무줄 연결점 제공
\end{itemize}

\subsection{2. 낙하산 시스템}

\textbf{기능}: 낙하 속도 감소로 충격 에너지 자체를 줄임

\begin{itemize}
    \item 크기: 직경 300-400 mm
    \item 재질: 비닐봉지 또는 얇은 천
    \item 부착 위치: 정육면체 상면 중앙
    \item 연결: 상면 4개 모서리에 실로 연결
    \item 역할: 공기 저항으로 터미널 속도 감소
\end{itemize}

\subsection{3. 고무줄 서스펜션 시스템}

\textbf{기능}: 12개 고무줄로 계란을 3차원 공간에서 중앙 서스펜션

\begin{itemize}
    \item 상면 연결: 4개 고무줄 (상면 4개 꼭짓점 → 계란 본체)
    \item 하면 연결: 4개 고무줄 (하면 4개 꼭짓점 → 계란 본체)
    \item 측면 연결: 4개 고무줄 (4개 측면 중심 → 계란 본체)
    \item 길이: 각 200mm (자연 상태)
    \item 재질: 중간 강도 고무줄
    \item 역할: 탄성 복원력으로 충격 에너지 흡수
\end{itemize}

\subsection{4. 계란 보호 본체}

\textbf{기능}: 계란 직접 보호 및 3단계 충격 완화

\textbf{1단계: 지점토 보호층 (5-8mm)}
\begin{itemize}
    \item 재질: 지점토 약 50g
    \item 특성: 소성 변형으로 충격 흡수
    \item 장점: 계란 형태에 정확히 맞춤
\end{itemize}

\textbf{2단계: 에어캡 보호층 (10-15mm)}
\begin{itemize}
    \item 재질: 버블랩 2-3겹
    \item 특성: 공기 쿠션 효과로 충격 분산
    \item 장점: 가볍고 효과적인 충격 흡수
\end{itemize}

\textbf{3단계: 솜 보호층 (20-30mm)}
\begin{itemize}
    \item 재질: 솜 또는 부드러운 완충재
    \item 특성: 최종 완충 및 미세 진동 흡수
    \item 용기: 대형 종이컵에 담아 형태 유지
\end{itemize}

% ============================================
\section{재료 목록}
% ============================================

\begin{center}
\begin{tabular}{lccc}
\toprule
\textbf{부품명} & \textbf{규격} & \textbf{수량} & \textbf{재질} \\
\midrule
정육면체 모서리 & 400 mm, 직경 5-8 mm & 12개 & 플라스틱/나무 막대 \\
연결 부품 & 꼭짓점용 & 8개 & 글루건/연결고리 \\
낙하산 & 직경 300-400 mm & 1개 & 비닐봉지/천 \\
고무줄 & 길이 200 mm & 12개 & 고무줄 \\
지점토 & 50g & 1세트 & 점토 \\
에어캡 & 폭 300mm, 1m & 1롤 & 버블랩 \\
솜 & - & 적량 & 완충재 \\
종이컵 & 대형 & 1개 & 종이 \\
연결 실 & - & 적량 & 실/끈 \\
\bottomrule
\end{tabular}
\end{center}

\vspace{0.5cm}

\textbf{참고사항}:
\begin{itemize}
    \item 고무줄은 너무 강하면 충격 전달, 너무 약하면 바닥 충돌 위험
    \item 낙하산 중앙에 작은 구멍을 뚫어 공기 배출구 만들기
    \item 지점토는 계란에 직접 감싸서 형태 고정
    \item 종이컵은 보호층을 담고 고무줄 연결점 역할
\end{itemize}

% ============================================
\section{조립 순서}
% ============================================

\subsection{단계 1: 정육면체 프레임 제작}

\begin{enumerate}
    \item 400 mm 막대 12개 준비
    \item 8개 꼭짓점에서 막대들을 연결하여 정육면체 형성
    \item 글루건으로 모든 접합부 고정
    \item 구조가 흔들리지 않는지 확인
    \item 각 꼭짓점과 측면 중심에 고무줄 연결점 표시
\end{enumerate}

\subsection{단계 2: 계란 보호 본체 제작}

\begin{enumerate}
    \item 계란에 지점토를 5-8mm 두께로 균등하게 감싸기
    \item 에어캡을 2-3겹으로 감싸기 (지점토 위에)
    \item 솜으로 최종 보호층 형성 (총 두께 40-50mm)
    \item 완성된 보호 계란을 종이컵에 넣어 형태 고정
    \item 종이컵 가장자리에 고무줄 연결 구멍 뚫기
\end{enumerate}

\subsection{단계 3: 고무줄 서스펜션 설치}

\begin{enumerate}
    \item 상면 4개 꼭짓점에서 계란 본체로 고무줄 4개 연결
    \item 하면 4개 꼭짓점에서 계란 본체로 고무줄 4개 연결
    \item 4개 측면 중심에서 계란 본체로 고무줄 4개 연결
    \item 계란이 정육면체 정중앙에 위치하도록 길이 조정
    \item 모든 고무줄의 장력이 균등한지 확인
\end{enumerate}

\textbf{주의}: 12개 고무줄이 모두 균등한 장력을 가져야 중앙 위치 유지

\subsection{단계 4: 낙하산 시스템 부착}

\begin{enumerate}
    \item 낙하산 중앙에 지름 3-5cm 구멍 뚫기 (공기 배출용)
    \item 상면 4개 모서리 중점에 실 연결
    \item 실의 다른 끝을 낙하산 가장자리에 균등하게 연결
    \item 낙하산이 펼쳐졌을 때 대칭인지 확인
    \item 낙하산 전개 테스트 실시
\end{enumerate}

\subsection{단계 5: 최종 검사 및 테스트}

\begin{enumerate}
    \item 모든 연결부 재점검
    \item 계란이 정중앙에 있는지 확인
    \item 고무줄 장력 균등성 재확인
    \item 낙하산 전개 상태 점검
    \item 낮은 높이에서 예비 테스트
\end{enumerate}

% ============================================
\section{충격 흡수 원리}
% ============================================

\subsection{4단계 충격 흡수 시스템}

\textbf{1단계: 낙하산 작동 (낙하 중)}
\begin{itemize}
    \item 공기 저항으로 터미널 속도 감소
    \item 낙하 속도 30-40\% 감소 → 충격 에너지 대폭 감소
    \item 에너지 = $\frac{1}{2}mv^2$ 이므로 속도 감소 효과 매우 큼
\end{itemize}

\textbf{2단계: 고무줄 서스펜션 작동 (착지 순간)}
\begin{itemize}
    \item 12개 고무줄이 동시에 늘어나며 충격 에너지 흡수
    \item 운동에너지 → 탄성에너지 변환: $E = \frac{1}{2}kx^2$
    \item 충격력의 40-50\% 흡수
    \item 탄성 복원력으로 2차, 3차 충격 방지
\end{itemize}

\textbf{3단계: 보호층 작동 (잔여 충격)}
\begin{itemize}
    \item 지점토: 소성 변형으로 에너지 흡수 (10-15\%)
    \item 에어캡: 공기 쿠션 효과로 충격 분산 (5-10\%)
    \item 솜: 최종 완충 및 미세 진동 흡수 (3-5\%)
\end{itemize}

\textbf{4단계: 종이컵 최종 보호}
\begin{itemize}
    \item 계란과 직접 접촉하는 최후 보호벽
    \item 형태 유지 및 보호층 고정 역할
\end{itemize}

\subsection{작동 흐름도}

\begin{center}
\fbox{
\begin{minipage}{0.9\textwidth}
\centering
낙하 시작
\\ $\downarrow$ \\
낙하산 전개 → 속도 감소 (30-40\%)
\\ $\downarrow$ \\
착지 순간: 12개 고무줄이 동시 작동 (40-50\%)
\\ $\downarrow$ \\
지점토 소성 변형 (10-15\%)
\\ $\downarrow$ \\
에어캡 공기 쿠션 효과 (5-10\%)
\\ $\downarrow$ \\
솜 최종 완충 (3-5\%)
\\ $\downarrow$ \\
\textbf{계란 안전!}
\end{minipage}
}
\end{center}

% ============================================
\section{예상 성능}
% ============================================

\begin{center}
\begin{tabular}{ll}
\toprule
\textbf{항목} & \textbf{예상 값} \\
\midrule
목표 낙하 높이 & 3-5 m \\
구조물 무게 & 425-585 g \\
전체 크기 & 약 600 mm $\times$ 600 mm $\times$ 600 mm \\
충격 흡수율 & 70-80\% (4단계 합산) \\
제작 시간 & 약 3-4시간 \\
예상 비용 & 7,000-12,000원 \\
\bottomrule
\end{tabular}
\end{center}

\subsection{성능 계산}

\textbf{총 충격 흡수율}:

각 단계의 흡수율을 $r_1, r_2, r_3, r_4$라 하면:
\begin{align}
\text{총 흡수율} &= 1 - (1-r_1)(1-r_2)(1-r_3)(1-r_4) \\
&= 1 - (1-0.35)(1-0.45)(1-0.125)(1-0.075) \\
&= 1 - 0.65 \times 0.55 \times 0.875 \times 0.925 \\
&= 1 - 0.290 = 71.0\%
\end{align}

\subsection{장점}

\begin{enumerate}
    \item \textbf{높은 충격 흡수율}: 4단계 시스템으로 70\% 이상 흡수
    \item \textbf{전방향 보호}: 12개 고무줄로 어떤 방향 충격도 대응
    \item \textbf{탄성 복원}: 고무줄이 원상복구되어 2차 충격도 흡수
    \item \textbf{재사용 가능}: 고무줄과 프레임은 재사용 가능
    \item \textbf{안정적 구조}: 정육면체로 구조적 안정성 확보
\end{enumerate}

\subsection{주의사항}

\begin{enumerate}
    \item 고무줄 장력이 불균등하면 계란이 한쪽으로 치우침
    \item 낙하산이 비대칭이면 회전하며 낙하할 위험
    \item 보호층이 너무 두꺼우면 무게 증가로 충격 에너지 증가
    \item 종이컵과 고무줄 연결이 약하면 분리 위험
\end{enumerate}

% ============================================
\section{수학적 계산}
% ============================================

\subsection{고무줄 탄성 에너지}

12개 고무줄의 탄성 계수를 $k$, 변형량을 $x$라 하면:

\textbf{총 탄성 에너지}:
\[
E_{elastic} = 12 \times \frac{1}{2}kx^2 = 6kx^2
\]

\textbf{충격 에너지 흡수}:
계란의 운동 에너지 $E_k = \frac{1}{2}mv^2$가 탄성 에너지로 변환

\subsection{낙하산 효과}

\textbf{터미널 속도}:
공기 저항 = 중력일 때의 속도
\[
v_{terminal} = \sqrt{\frac{2mg}{\rho A C_d}}
\]

여기서:
\begin{itemize}
    \item $m$: 구조물 질량
    \item $g$: 중력 가속도 (9.8 m/s²)
    \item $\rho$: 공기 밀도 (1.225 kg/m³)
    \item $A$: 낙하산 면적
    \item $C_d$: 항력 계수 (약 1.3-1.5)
\end{itemize}

\textbf{낙하산 면적}:
직경 350mm 낙하산의 경우
\[
A = \pi r^2 = \pi \times (0.175)^2 \approx 0.096 \text{ m}^2
\]

% ============================================
\section{참고 자료}
% ============================================

\subsection{관련 물리 원리}

\begin{itemize}
    \item \textbf{후크의 법칙}: $F = kx$ (고무줄 탄성력)
    \item \textbf{에너지 보존}: 운동에너지 ↔ 탄성에너지 변환
    \item \textbf{공기 저항}: $F_d = \frac{1}{2}\rho v^2 C_d A$ (낙하산 효과)
    \item \textbf{충격량-운동량 정리}: 충격 시간 증가 → 충격력 감소
\end{itemize}

\subsection{개선 아이디어}

\begin{enumerate}
    \item 고무줄 대신 스프링 사용 → 더 정확한 탄성 계수
    \item 댐핑 기능 추가 → 진동 감소
    \item 낙하산 형태 최적화 → 항력 계수 향상
    \item 보호층 재료 변경 → 무게 대비 성능 향상
\end{enumerate}

\vspace{1cm}

\begin{center}
\large
\textbf{--- 자유설계 설계서 끝 ---}
\end{center}

\end{CJK}
\end{document}
